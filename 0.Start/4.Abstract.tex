\begin{abstract}

Flows of solid-fluid mixtures pose persistent challenges to our conceptualization, experimentation and modeling capabilities. The wide array of possibly relevant scales, both spatial and temporal, have prevented the design of models capable of providing useful solutions for complex, multi-scale situations. Competent models are applied to very specific scales and types of flow, facing severe problems out of their narrow domain of application.

Direct observation and measurement of many of these flows quantities are difficult or impossible to achieve, justifying the limited amount of literature on the subject of more robust conceptual models. With increasing computational capabilities, there is hope that highly resolved models with small number of assumptions can lead to approximate solutions for these flows. This has the potential to impose new research prompts, leading to better understandings of the phenomena.

The key objective of this dissertation is to introduce a unified discretisation of rigid solids and fluids, allowing for resolved simulations of fluid-solid phases within a meshless framework. The numerical solution, attained by Smoothed Particle Hydrodynamics (SPH) and a variation of Discrete Element Method (DEM), the Distributed Contact Discrete Element Method (DCDEM) discretisations, is achieved by directly considering solid-solid and solid-fluid interactions. The novelty of the work is centered on the generalization of the coupling of the DEM and SPH methodologies for resolved simulations, allowing for state-of-the-art contact mechanics theories to be used in arbitrary geometries, while fluid to solid and vice versa momentum transfers are accurately described. The methods are introduced, analyzed and discussed. 

A series of experimental campaigns are devised to serve as validation for complex solid-fluid flows simulations and together with analytical and other benchmark numerical solutions, allow for a comprehensive characterization of the model. Unique experiments were performed, such as dam-break flow with movable objects and settling dynamics of macroscopic solid particles. For the dam-break tests, a set of blocks is placed in several configurations and then subjected to the bore and subsequent unsteady flow. Blocks are tracked and positions are then compared between experimental data and the numerical solutions. A PIV technique allows for the quantification of the flow field and direct comparison with numerical data. The results show that the model is accurate and is capable of treating highly complex interactions, such as transport of debris or unsteady hydrodynamic actions on structures, if relevant scales are reproduced. The settling case allowed to guarantee that relevant hydrodynamic forces are correctly modeled.

Preliminary results in limit cases are presented and discussed. These are cases whose numerical treatment has proven challenging for other models and experimental initiatives are either expensive or limited in the amount of data that can be extracted.

\end{abstract}