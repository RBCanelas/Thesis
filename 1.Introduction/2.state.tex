\section{Objectives and Structure of the Dissertation}
\label{sec:objectives}

The range of relevant spatial and time scales, coupled with potentially very complex geometries, has prevented the appearance of generalized models, that can claim to provide solutions for very different problems in the realm of solid-fluid flows. The main ambition of this dissertation is the development of a model capable of providing meaningful solutions to such problems. The scope of the work are problems of fluvial hydraulics such as debris flows, up to large scale industrial and urban problems such as large debris transport by fluid action. The key objectives are then to propose a model that is able to cope with truly arbitrary geometries, distinct bodies, of distinct materials interacting in a highly non-linear fashion with a highly unsteady, discontinuous flow, with relevant scales ranging from few centimeters to tens of meters. The proposed dissertation work draws inspiration from the seminal works of \cite{Koshizuka-1998} and \cite{Potapov-2001}, effectively combining a general form of \ac{DEM}, \ac{DCDEM}, with an \ac{SPH} formulation.

The basic tools to ensure the completion of the objectives are coincident with that of any work that attempts at mathematical modeling of any phenomena in any specific framework: i) the design of a sound conceptual model, composed of system of conservation equations, as well as all necessary closure equations, if the system is open; ii) the discretization of the system and the assembly of an appropriate numerical scheme and iii) validation of the model with the application to documented case studies, that were not used as a phenomenological basis for the conceptual model. A largely neglected component of exploratory numerical studies is the quality of the implementation and the adaptation of the devised scheme to \ac{HPC} architectures. One of the fundamental premises of this work was precisely the possibility of use of the model under realistic conditions. Thus adding to the points previously presented, implementation of the numerical scheme using state-of-the-art techniques to ensure simultaneous readability, modularity, expandability and maximum performance of the code is a paramount objective.

Under these general guidelines, the detailed objectives are presented simultaneously as the structure of the thesis, so structured to reinforce the several steps that compose the work.

Chapter \ref{cap:conceptual}, dealing with the derivation of conceptual models that will serve as a basis for numerical discretization, has the objective of introducing a coherent language, both at the notation and phenomenological levels. Continuum descriptions are derived and an almost complete form of the final conceptual model is presented. Some of the phenomenology treated in this chapter, namely contact laws for solid-solid interactions, are a conundrum, at best. They correspond to a higher form of educated guesses, based on limited data on hard to observe microscopic events, but trivial to record macroscopic effects. The conceptual model is promoted in Chapter  \ref{cap:conceptual}, but the final form of the contact laws only appears in Chapter \ref{cap:chapter-numerical}, as they are designed in discrete form.

Chapter \ref{cap:chapter-numerical} is devoted to introducing and analyzing fundamental aspects of the discretization methods, \ac{SPH} and \ac{DEM}. Known difficulties are explored in order to allow for an easier reading of the following chapters. The discrete operators of the \ac{SPH} method are derived for generic problems, and applied to different phases trough the sections of the chapter. A rigid body formulation is introduced, enabling the coexistence of solid and fluid particles in the same solution. Further discussion of the contact laws that provide contact force estimates for the \ac{DEM} model takes place. Numerical stability is discussed in an attempt to enforce a correct description of all involved time scales, on both models. The formulation roughly corresponds to the initial iterations of the model, first presented in \cite{Canelas-al-2013b} and \cite{Canelas-al-2013c}. \cite{Canelas-al-2015a} and \cite{Canelas-al-2015b} provide a complete reference to the model. Portraying to the \ac{SPH} model alone, the work follows \cite{Crespo-2015} closely, since this dissertation feeds into the open source code DualSPHysics, developed in colaboration between teams at the Univerity of Vigo, University of Manchester and Instituto Superior T\'{e}cnico.

Chapter \ref{cap:chapter_hpc} has the objective of addressing important implementation details. The algorithmic structure of the method is analyzed and used to expose possible parallelism and predictable bottlenecks. A superficial introduction to the structure of the code in its \ac{CPU} and \ac{GPU} manifestations is carried out, with special care to point out major differences and particular adaptations to the particularities of the architectures, loosely adhering to the work in \cite{Canelas-al-2013b} and also explored in \cite{Crespo-2015}. This chapter serves as a warning: performance comes at a cost, greatly deductible if the design process is integrated. The numerical scheme should be written in a way to maximize computability for a given computer architecture, at the risk of voting to irrelevance an otherwise successful or even revolutionary approach.

In Chapter \ref{cap:chapter_validation}, one of the fundamental requirements of mathematical modeling is fulfilled. The model is compared against known solutions for both canonical problems and more subtle cases. Analytical solutions, reference numerical results and original experimental campaigns were used to provide a broad spectrum testing program of the proposed model. \cite{Canelas-al-2015a} focused on fluid-solid interactions, corresponding to sections \ref{Subsect:free_stream} and \ref{sec:validation_buoyancy}. Sections \ref{sec:validation_exp} and \ref{Subsect:PIV} follow the structure of the solid-solid and fluid-solid interaction validations presented in \cite{Canelas-al-2015b}.

As the model is intended to provide data on scenarios currently inaccessible by other means, Chapter \ref{cap:chapter_apps} provides initial results on such three cases. The large scale harbor case presented in section \ref{sec:sines} draws from the work presented in \cite{Canelas-al-2014}.

Chapter \ref{cap:conclusions} draws global conclusions, comments on the results of each individual chapter and provides a small series of recommendations for future developments.\\

List of published related works

\begin{itemize}
\item Canelas, R.B., Crespo, A.J.C., Dom\'{i}nguez, J.M., G\'{o}mez-Gesteira, M., and Ferreira, R.M.L. "SPH-DCDEM model for arbitrary geometries in free surface solid-fluid flows." Computer Physics Communications (2015) Submitted.
\item Canelas, R.B., Dom\'{i}nguez, J.M., Crespo, A.J.C., G\'{o}mez-Gesteira, M., and Ferreira, R.M.L. "A Smooth Particle Hydrodynamics discretization for the modelling of free surface flows and rigid body dynamics." International Journal for Numerical Methods in Fluids (2015), 78, 581-593.
\item  Crespo, A.J.C., J.M. Dom\'{i}nguez, B.D. Rogers, M. G\'{o}mez-Gesteira, S. Longshaw, R.B. Canelas, R. Vacondio, A. Barreiro, and O. Garc\'{i}a-Feal. "DualSPHysics: Open-source parallel CFD solver based on Smoothed Particle Hydrodynamics (SPH)." Computer Physics Communications 187 (2015): 204-216.
\item Canelas, R.B., Aleixo, R., Ferreira, R.M.L. "SPH-based numerical simulation of the velocity field in a dam-break flow." In MEFTE 2012, APMTAC (ed), Lisbon
\item Canelas, R.B., R.M.L. Ferreira, A.J.C. Crespo, and J.M. Dom\'{i}nguez. "A generalized SPH-DEM discretization for the modelling of complex multiphasic free surface flows." In the 8th international SPHERIC workshop. 2013.
\item Canelas, R.B., Ferreira, R.M.L., Dom\'{i}nguez, J.M. and Crespo, A.J.C. "Modelling Of Wave Impacts On Harbour Structures And Objects With SPH And DEM." In the 9th international SPHERIC workshop. 2014.
\item Canelas, R.B., J.M. Dom\'{i}nguez, and R.M.L. Ferreira. "Coupling a Generalized DEM and an SPH Models Under a Heterogeneous Massively Parallel Framework." In CMN 2013, Bilbao, SEMNI (ed) (2013).
\item  Canelas, R.B., R.M.L. Ferreira, A.J.C. Crespo, and J.M. Dom\'{i}nguez. "Numerical modeling of complex solid-fluid flows with meshless methods." RiverFlow 2014 (2014)
\end{itemize}
