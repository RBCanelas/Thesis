\section{Mathematical Modeling of Solid-Fluid Flows, an Overview}
\label{sec:int_motivation}

Flows of solid-fluid mixtures cover a large spectrum of scales: from particles so small that its kinematics are dominated by random molecular motion, to groups of kilometre-wide icebergs being dragged by oceanic currents, in geophysical settings. Such flows, however, present a series of challenges for both experimental and numerical research. In an experimental campaign, efficient techniques used in single-phase measurements such as hot-wire anemometry, Laser Doppler Anemometry (LDA) or \ac{PIV}, only provide accurate measures of the velocity field if the flows are dilute and the particles are relatively small. Measuring solid concentration and local variations is equally problematic. As only recently non-intrusive techniques, such as Nuclear Magnetic Resonance \citep{Fukushima-1999, Lemonnier-2010}, are being explored to study these flows, optical methods continue to be the main approaches \citep{Douxchamps-2002, Armanini-2008}, again with serious difficulties for dense and highly three-dimensional flows \citep{Spinewine-2003}. 

In a similar fashion, numerical simulations of solid-liquid flows are demanding because of the complex geometries and the types of momentum transfer modes that arise from the fluid-solid interactions. %Although the phenomena on all of this scale spectrum are described by the same dynamics laws, the very scale may introduce important simplifications in order to devise an efficient predictive model, from an engineering standpoint. For this reason, and due to chronically limited computing capabilities, numerical model usually rely on conceptual models assembled for a specific range of space and time scales. This allows the disregard of phenomenological considerations that manifest orders of magnitude 
Accurate and numerically efficient simulations are of substantial importance for research and industrial fields, as they allow to derive many important constitutive relations and further research prompts. A special success story, close to the current topic would be dry granular flows. Considering mainly spherical particles, interactions are easier to approximate, resulting in a large body of work being produced \citep{Campbell-2006}, with profound implications in the industry.

Due to the difficulties associated with the modeling of solid-fluid flows, simulations are usually constrained to a relatively small range of the scale spectrum. Several proposals for unresolved models were presented, both in coupled and uncoupled versions \citep{Calantoni-2004, Robinson-2014, Cleary-2014}. These rely on the parametrization of the bulk solid-fluid interactions. Resolved models are also typically scale specific, as the solid fraction is generally described by spherical \ac{DEM} particles \citep{Potapov-2001, Kempe-2012} and special considerations to account for contact lubrication are derived from small scale experimental studies, such as the works of \cite{Yang-2006} and \cite{Joseph-2001}. 

The model presented in this thesis was designed to accommodate the concerns common in several technical disciplines, including coastal, offshore, maritime and a large part of fluvial engineering. In these disciplines, the common simplification that the solid material is perfectly rigid allows for robust solutions to a large array of problems. A computational model capable of providing meaningful solutions for the interaction of fluid and rigid solid objects is a valuable tool for the quantification of severity of hydrodynamic actions in several contexts, including risk assessment studies, design of floating bodies or design of exposed structures. Such tool must be computationally scalable, should be able to model all physically relevant scales which fluid-solid interaction occurs. In most engineering applications involving fluids and structures, solid objects are much larger than the smallest flow scales. For instance, viscous modes of momentum transfer are often negligible since the involved Reynolds numbers are normally large \citep{Shu-2011}. However, the relevant modes of interaction are not always evident, in which case the model must be designed to offer high spatial and temporal resolutions. Also, to minimize the influence of imposed non-physical boundaries (lateral walls or periodic zones in an open beach for example), some simulations require remarkably large domains. This highlights the need for high performance models and implementations. Finally, such models should be based on consistent conceptual models, i.e. systems of conservation equations and closure equations, avoiding \emph{ad hoc} formulations, and should be subjected to a discretization that preserves the key mathematical properties of the conceptual model. All of the required characteristics point to the need for a resolved model, in order to cope with complex geometries and the range of potentially important spatial and temporal scales.

Three-dimensional, fully coupled and interface-resolving simulations of flows with arbitrary numbers of solid particles have attracted considerable attention from the academic environment. Within the mesh-based ideas, the \ac{IBM}, as originally proposed by \cite{Peskin-1977}, has arguably been the most adapted \citep{Prosperetti-2007}. The basic idea of this approach is to employ a mesh for the discretization of the fluid phase and to represent the immersed fluid-solid interface by surface markers. In order to satisfy the required boundary conditions at the interface additional source terms are used in the momentum equation. \cite{Fekken-2004} coupled a \ac{VOF} method to a quarterion solver by using a modified \ac{IBM} version. It allowed for complex flows with simple geometries to be studied, including the effects of the free-surface, but no solid-solid considerations were made. The main problem with the meshed approach is the growing numerical and computational complexity with growing scene complexity. This imposes hard limits on the applicability of models that are not tailor tuned to a specific application, as optimization as data management become increasingly difficult.

Within the meshless framework, efforts have been made on unifying solid and fluid modeling. \cite{Koshizuka-1998} modeled a rigid body as a collection of Moving Particle Simulation (MPS) fluid particles that keep their relative distance by default. This has become the standard approach due to its simplicity and elegance. \cite{Monaghan-2003} and \cite{Rogers-2010}, employing the same principle, modeled the effects of wave interaction on rigid bodies resorting to \ac{SPH} and special considerations for the particles that belonged to the solid body, effectively including a form of frictional behavior. In his work, \cite{Potapov-2001} used a standard \ac{DEM} formulation to treat the solid phase, employing contact mechanics formulations. The fluid phase is treated with a standard \ac{SPH} model, treating the interface between solid and fluid with a ghost particle method. This allows to interpolate the pressure and drag forces to the solid body, resolving the interface to the desired scale. Even limited to spherical solid bodies, the model presented unique results concerning neutrally buoyant particles contained between two plates for different solid fractions, fluid viscosities and shear rates, reproducing results from the Bagnold experiments \citep{bagnold-1954}. 

Effective blends of meshed and meshless methods have been explored recently, such as \ac{PIC} methods and \ac{PFEM} \citep{Liu-2003}. \ac{PIC} methods have been used since the 50s, mostly in plasma and other high energy physics \citep{Evans-1957}. It poses the same difficulties as most of the mesh based models, as Lagrangian particles are in fact being advected on Eulerian fields. \ac{PFEM} uses much of the \ac{FEM} formulation, with nodes that are moved at every time step, with a constant need for re-meshing \citep{Onate-al-2004, Idelsohn-al-2004}. This imposes great computational complexity and limits the mapping of the code to massively parallel architectures, while allowing for seamless integration with decades \ac{FEM} research and development.
